\documentclass{article}
\usepackage{kotex, geometry, hyperref, graphicx}

\title{Week 9 - HMI Research Group \\ \large 31 Jul 2017 - 4 Aug 2017}
\author{Long Phi Nguyen (뉀피롱), 한국전자통신연구원}

\begin{document}

  \pagenumbering{gobble}
  \maketitle


  \subsection*{Summary} Speech module is almost at its final stages. Integration with gesture generation remains.

  \subsection*{Points}
  \begin{itemize}
    \item Using \verb|spaCy|, wrote \verb|language_processing.py|, which is part of the entire workflow (composed of \verb|language_processing.py|, \verb|gesture_suite.py|, and the \verb|pickles| folder, where all file data is stored).
    \item Defined a preliminary corpus with roughly 10 entries defined for \verb|yes| and \verb|no|.
    \item Trained an online model, \verb|SGDClassifier| from \verb|sklearn|, to classify new inputs and learn on the fly and adjust itself. Test results are good so far.
  \end{itemize}

  \subsection*{Plans}
  \begin{itemize}
    \item Feature more classes in the classification problem. Ideas in mind include \verb|you|, \verb|this|, and more.
    \item Consult an online word/sentence bank or something similar to train the machine learning model with much more than just 10 examples.
    \item Completely integrate the speech and gesture modules together to get the final product.
    \item Transfer experiments over to the real NAO robot.
  \end{itemize}

  \subsection*{Addendum}
  The repository can be found \href{https://github.com/longnguyen1997/nao_animations}{\texttt{here}}. The new file, \verb|language_processing.py| is included. Another new file, \verb|corpus_script.py|, was used to construct the corpus and dump it to disk for the speech module to read.

\end{document}
