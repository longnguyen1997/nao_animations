\documentclass{article}
\usepackage{kotex, geometry, hyperref, graphicx}

\title{Week 5 - HMI Research Group \\ \large 3 Jul 2017 - 7 Jul 2017}
\author{Long Phi Nguyen (뉀피롱), 한국전자통신연구원}

\begin{document}

  \pagenumbering{gobble}
  \maketitle


  \subsection*{Summary} A new Python script/class was created to analyze motion from NAO's sensors. New, natural movements
  have been successfully generated.

  \subsection*{Points}
  \begin{itemize}
    \item Started over and collected 10,000+ sensor data reports from NAO (running on \verb|webots|)---this data
    included \verb|BodyTalk| gestures, which are meant to be used when NAO speaks casually, and \verb|Emotions/Negative|
    gestures.
    \item Wrote a class that uses \verb|numpy|, \verb|seaborn|, and \verb|naoqi| together to provide analytics and
    functional behavior for NAO's sensor data. This class plots sensor values with a kernel density estimate to indicate
    joint positions that NAO often has when performing certain types of gestures. By sampling random values close to the
    mean ($\pm$ standard dev.) of NAO's joints, it generates new motions.
    \item Generated motion works beautifully for NAO (\verb|BodyTalk| category). Motions are smooth, and each is entirely new.
  \end{itemize}

  \subsection*{Plans}
  \begin{itemize}
    \item Clashing sensor distribution data from \verb|Emotions/Negative| gestures causes NAO to fall down when
    generating new movements---need to fix this; new motions sampled from \verb|BodyTalk| are fine.
    \item Analyze all the other subsets of gestures---currently only done with about $50/600+$ gestures.
  \end{itemize}

  \subsection*{Addendum}
  Again, all progress made can be found \href{https://github.com/longnguyen1997/nao_animations}{\texttt{here}}.
  Plots from the sensor data analysis can be found \href{https://github.com/longnguyen1997/nao_animations/tree/master/plots}{\texttt{here}}.



\end{document}
