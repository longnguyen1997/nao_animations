\documentclass{article}
\usepackage{kotex, geometry, hyperref, graphicx}

\title{Week 7 - HMI Research Group \\ \large 17 Jul 2017 - 21 Jul 2017}
\author{Long Phi Nguyen (뉀피롱), 한국전자통신연구원}

\begin{document}

  \pagenumbering{gobble}
  \maketitle


  \subsection*{Summary} A \emph{huge} bug was found when going through the codebase this week: this whole time, wrong sensor names were being assigned to angle/data points. NAO only fell down because of this, and it explains weird animations being triggered when trying to replicate motion with collected data.

  \subsection*{Points}
  \begin{itemize}
    \item Reread the low-dimensional embedding paper and experimented with Python package \verb|GPy| (Gaussian process modeling package).
    \item Reparsed sensor and data correspondence and finally debugged NAO falling down.
    \item Wrote an $O(n)$ algorithm incorporated into sensor data collection to bridge data points by intervals of 0.3s. These points can then all be interpolated sequentially with 0.3s transitions to replicate the original movements.
    \item Tested original method of statistical sampling within the mean by a standard deviation---movements are fine now and replicate original Aldebaran gestures well, but with their own exaggerations for natural effect.
  \end{itemize}

  \subsection*{Plans}
  \begin{itemize}
    \item Need to determine a way to sample a trajectory and map it back to 26-dimensional space to generate new motions while preserving old trajectory sequences.
    \item Refactor \verb|nao_script.py| and \verb|motion_analyzer.py| together into a new, modular class, preferably derived from \verb|naoqi|'s \verb|ALModule| class.
  \end{itemize}

  \subsection*{Addendum}
  The repository can be found \href{https://github.com/longnguyen1997/nao_animations}{\texttt{here}}.
  Time series plots for gestures from \verb|BodyTalk| were redone with the correct sensor relations and can be found
  \href{https://github.com/longnguyen1997/nao_animations/tree/master/plots/time_series/standing_bodytalk}{\texttt{here}}.

\end{document}
